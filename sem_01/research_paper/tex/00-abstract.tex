\begin{essay}{}
    \noindent\textbf{Ключевые слова}: цифровое изображение, суперрезолюция, суперразрешение, регуляризация, интерполяция, теория множеств, глубокое обучение, сверточные нейронные сети.

    Объектом исследования является повышение разрешения изображения по нескольким кадрам низкого разрешения. Предметом исследования является набор цифровых изображений одного и того же объекта.
    
    Целью работы являлась классификация известных методов повышения разрешения изображения по нескольким кадрам.

    Для достижения поставленной цели были выполнены следующие задачи:

    \begin{itemize}
        \item проведен обзор существующих методов повышения разрешения изображения по нескольким кадрам;
        \item сформулированы критерии классификации и сравнения методов;
        \item классифицированы рассмотренные методы;
        \item проведен сравнительный анализ рассмотренных методов;
        \item на основе полученных теоретических сведений сделаны вывод об области применимости методов.
    \end{itemize}

    В результате было установлено, что выбор метода суперрезолюции зависит от конкретной задачи и имеющихся вычислительных и временных ресурсов. Методы на основе глубокого обучения часто дают наилучшие результаты, но они требуют большого объема данных и мощных вычислительных ресурсов для обучения сетей. Аналитические методы более вычислительно эффективны, но могут не давать такие высококачественные результаты, а также не учитывать структурные особенности изображений.

    На основе проведенной классификации и сравнительного анализа для дальнейшей разработки было выбрано направление оптимизации сверточных нейронных сетей для повышения разрешения изображения по нескольким кадрам.
    
\end{essay}