\chapter*{ВВЕДЕНИЕ}
\addcontentsline{toc}{chapter}{ВВЕДЕНИЕ}

Среди способов восприятия человеком информации об окружающем мире посредством органов чувств зрение занимает особое место --- с помощью глаз в среднем воспринимается до 80~\% информации, поступающей из внешней среды~\cite{percent}. Именно поэтому зрительные образы, часто запечатляемые снимками фотокамеры, играют важнейшую роль в нашей жизни.

Для многих задач, связанных с областью цифровой обработки сигналов, высокое разрешение изображения является ключевым аспектом, позволяющим проводить более качественный анализ и последующую обработку полученной информации. Изображение высокого разрешения содержит больше сведений о деталях, учет которых может быть критически важен во многих областях, таких как медицина, астрономия и многие другие.

В процессе формирования изображения качество может ухудшиться по многим причинам, таким как размытие, дефокусировка, движение камеры, шум, вибрации, малое количество сенсоров фотокамеры. Одним из способов решения рассматриваемой проблемы является использование более качественной техники, однако при таком подходе быстро достигается лимит стоимости, веса и размеров оборудования, что делает этот подход нецелесообразным с точки зрения использования ресурсов. Также понижение качества часто может быть связано с физическими ограничениями (например, при регистрации спутников).

В связи с этим существует широкое разнообразие методов повышения разрешения изображения, систематизация которых является актуальной задачей для систем фотосъемки.