\chapter*{ЗАКЛЮЧЕНИЕ}
\addcontentsline{toc}{chapter}{ЗАКЛЮЧЕНИЕ} 

В данной работе была рассмотрена задача повышения разрешения изображения по нескольким кадрам (задача суперрезолюции).

Для достижения поставленной цели были рассмотрены существующие методы, а также проведены классификация и сравнительный анализ методов.

Были рассмотрены следующие группы методов: частотные методы, методы на основе интерполяции, методы на основе теории множеств, методы на основе решения обратной задачи с применением регуляризации, примеро~--~ориентированные методы, методы с использованием сверточных нейронных сетей.

Выбор метода суперрезолюции зависит от конкретной задачи и имеющихся вычислительных и временных ресурсов. Методы на основе глубокого обучения часто дают наилучшие результаты, но они требуют большого объема данных и мощных вычислительных ресурсов для обучения сетей. Аналитические методы более вычислительно эффективны, но могут не давать такие высококачественные результаты, а также не учитывать структурные особенности изображений.

На основе проведенной классификации и сравнительного анализа для дальнейшей разработки было выбрано направление оптимизации сверточных нейронных сетей для повышения разрешения изображения по нескольким кадрам, т.к. в рамках поставленной задачи ключевым фактором является качество получаемого результата, другие же факторы (вычислительная сложность, необходимость пост~--~или предобработки) являются второстепенными.

