\addcontentsline{toc}{chapter}{СПИСОК ИСПОЛЬЗОВАННЫХ ИСТОЧНИКОВ}
\renewcommand\bibname{СПИСОК ИСПОЛЬЗОВАННЫХ ИСТОЧНИКОВ}
\bibliographystyle{utf8gost705u}  % стилевой файл для оформления по ГОСТу
\begin{thebibliography}{3}
    \makeatletter
    \def\@biblabel#1{#1. }

    \bibitem{percent}
    Раков Д. Л. Морфологический анализ и синтез невозможных объектов //IMPART-2014. Невозможные объекты и оптические иллюзии в современном искусстве и дизайне (Традиционные и компьютерные технологии). – 2014. – С. 16-19.
    
    \bibitem{digital_pic_1}
    Гришенцев А. Ю., Коробейников А. Г. Методы и модели цифровой обработки изображений. – 2014.

    \bibitem{digital_pic_2}
    Гонсалес Р., Вудс Р. Цифровая обработка изображений. – Litres, 2022.

     \bibitem{n1}
    Manaila F., Boiangiu C. A., Bucur I. Super resolution from multiple low resolution images //Journal Of Information Systems & Operations Management. – 2014. – С. 1.

    \bibitem{n2}
    Glasner D., Bagon S., Irani M. Super-resolution from a single image //2009 IEEE 12th international conference on computer vision. – IEEE, 2009. – С. 349-356.

    \bibitem{n3}
    Farsiu S. et al. Advances and challenges in super‐resolution //International Journal of Imaging Systems and Technology. – 2004. – Т. 14. – №. 2. – С. 47-57.

    \bibitem{model}
    Насонов А. В., Крылов А. С. Быстрое суперразрешение изображений с использованием взвешенной медианной фильтрации //Труды. – 2010. – С. 101-104.

    \bibitem{frequency}
    Park S. C., Park M. K., Kang M. G. Super-resolution image reconstruction: a technical overview //IEEE signal processing magazine. – 2003. – Т. 20. – №. 3. – С. 21-36.

    \bibitem{p}
    Кокошкин А. В. и др. Оценка ошибок синтеза изображений с суперразрешением на основе использования нескольких кадров //Компьютерная оптика. – 2017. – Т. 41. – №. 5. – С. 701-711.

    \bibitem{patch}
    Wang Q., Tang X., Shum H. Patch based blind image super resolution //Tenth IEEE International Conference on Computer Vision (ICCV'05) Volume 1. – IEEE, 2005. – Т. 1. – С. 709-716.

    \bibitem{example}
    Freeman W. T., Jones T. R., Pasztor E. C. Example-based super-resolution //IEEE Computer graphics and Applications. – 2002. – Т. 22. – №. 2. – С. 56-65.

    \bibitem{mnk}
    Katsaggelos A. K. Digital image restoration. – Springer Publishing Company, Incorporated, 2012.

    \bibitem{map}
    Elad M., Feuer A. Restoration of a single superresolution image from several blurred, noisy, and undersampled measured images //IEEE transactions on image processing. – 1997. – Т. 6. – №. 12. – С. 1646-1658.

    \bibitem{pocs}
    Fan C. et al. POCS Super-resolution sequence image reconstruction based on improvement approach of Keren registration method //Sixth International Conference on Intelligent Systems Design and Applications. – IEEE, 2006. – Т. 2. – С. 333-337.
    
    \bibitem{pocs2}
    Stark H., Oskoui P. High-resolution image recovery from image-plane arrays, using convex projections //JOSA A. – 1989. – Т. 6. – №. 11. – С. 1715-1726.
    
    \bibitem{cnn}
    Бредихин А. И. Алгоритмы обучения сверточных нейронных сетей //Вестник Югорского государственного университета. – 2019. – №. 1 (52). – С. 41-54.

    \bibitem{cnn2}
    Umehara K., Ota J., Ishida T. Application of super-resolution convolutional neural network for enhancing image resolution in chest CT //Journal of digital imaging. – 2018. – Т. 31. – С. 441-450.

    \bibitem{cnn3}
    Dong C. et al. Learning a deep convolutional network for image super-resolution //Computer Vision–ECCV 2014: 13th European Conference, Zurich, Switzerland, September 6-12, 2014, Proceedings, Part IV 13. – Springer International Publishing, 2014. – С. 184-199.

    \bibitem{pav}
    Senov A. Projective approximation based quasi-Newton methods// In: Proc. of International Workshop on Machine Learning, Optimization, and Big Data. 2017. P. 29–40.
    \bibitem{senov}
    Сенов А. А. Глубокое обучение в задаче реконструкции суперразрешения изображений //Стохастическая оптимизация в информатике. – 2017. – Т. 13. – №. 2. – С. 38-57.
    
\end{thebibliography}
