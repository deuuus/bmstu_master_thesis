\begin{essay}{}
    \noindent\textbf{Ключевые слова}: цифровое изображение, суперразрешение, свертка, сверточные нейронные сети, контрастно~--~ограниченная эквализация гистограмм.

    Объектом исследования является повышение разрешения изображения по нескольким кадрам низкого разрешения.
    
    Предметом исследования является разработка модификации классической архитектуры сверточной нейронной сети для решения поставленной задачи.

    Целью работы являлась разработка метода повышения разрешения изображения по нескольким кадрам.

    Для достижения поставленной цели были выполнены следующие задачи:

    \begin{itemize}
        \item[---] изложены особенности предлагаемого метода;
        \item[---] описаны основные этапы разрабатываемого метода в виде детализированной диаграммы IDEF0 и схем алгоритмов;
        \item[---] спроектирована структура программного обеспечения для реализации разрабатываемого метода.
    \end{itemize}

    В результате была сформулирована архитектура модифицированной сверточной нейронной сети: на первом этапе работы нейронной сети применяеются несколько параллельных сверточных слоев с возможным уплотнением карт признаков для каждого входного изображения, затем полученные карты объединяются. Полученные карты признаков передаются в остаточные слои (англ. Residual Blocks) с целью проведения более эффективного обучения сети. В выходном слое к каждой карте также применяется свертка с нелинейной функцией активации, формируя итоговое изображение.
    
    На основе проведенной работы также определено направление дальнейших исследований метода, заключающееся в использовании различных способов объединения карт признаков входных изображений.
\end{essay}