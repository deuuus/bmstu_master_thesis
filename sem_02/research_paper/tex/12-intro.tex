\chapter*{ВВЕДЕНИЕ}
\addcontentsline{toc}{chapter}{ВВЕДЕНИЕ}

Существует множество способов повышения разрешения по нескольким кадрам, таких как: частотные методы, методы на основе интерполяции, методы на основе теории множеств, методы на основе решения обратной задачи с применением регуляризации, примеро~--~ориентированные методы, методы с использованием сверточных нейронных сетей \cite{frequency, example, patch, cnn, map, mnk, pocs, pav, senov}.

Выбор метода суперразрешения зависит от конкретной задачи и имеющихся вычислительных и временных ресурсов. Методы на основе глубокого обучения часто дают наилучшие результаты, но они требуют большого объема данных и мощных вычислительных ресурсов для обучения сетей. Аналитические методы более вычислительно эффективны, но могут не давать такие высококачественные результаты, а также не учитывать структурные особенности изображений.

На основе проведенной классификации и сравнительного анализа для дальнейшей разработки было выбрано направление оптимизации сверточных нейронных сетей для повышения разрешения изображения по нескольким кадрам, т.к. в рамках поставленной задачи ключевым фактором является качество получаемого результата, другие же факторы (вычислительная сложность, необходимость пост~--~или предобработки) являются второстепенными.

Целью работы является разработка метода повышения разрешения изображения по нескольким кадрам.

Для достижения поставленной цели необходимо выполнить следующие задачи:

\begin{itemize}
    \item[---] изложить особенности предлагаемого метода;
    \item[---] описать основные этапы разрабатываемого метода в виде детализированной диаграммы IDEF0 и схем алгоритмов;
    \item[---] спроектировать структуру программного обеспечения для реализации разрабатываемого метода.
\end{itemize}