\chapter*{ЗАКЛЮЧЕНИЕ}
\addcontentsline{toc}{chapter}{ЗАКЛЮЧЕНИЕ} 

В данной работе была рассмотрена задача разработки метода повышения разрешения изображения по нескольким кадрам разной интенсивности.

Для достижения поставленной цели была формализована постановка задачи в виде диаграммы IDEF0 верхнего уровня, описаны основные этапы разрабатываемого метода в виде детализированной диаграммы IDEF0 и схем алгоритмов. Была спроектирована структура программного обеспечения для реализации разрабатываемого метода. 

Для оценки качества полученного результата предварительно были выбраны метрики MSE, PSNR, SSIM (и модификации), HPM.

Также были сформулированы требования и ограничения к разрабатываемому методу, требования к разрабатываемому программному обеспечению. Изложены особенности предлагаемого метода.

На основе проведенной работы была сформулирована первичная архитектура нейронной сети, решающей задачу суперразрешения. В рамках дальнейшего развития предполагается реализация предложенного метода и сопутствующая корректировка архитектуры сети, а также выработка подхода к обучению для разрабатываемой архитектуры в соответствии с полученными практическими результатами.