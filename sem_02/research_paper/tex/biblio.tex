\addcontentsline{toc}{chapter}{СПИСОК ИСПОЛЬЗОВАННЫХ ИСТОЧНИКОВ}
\renewcommand\bibname{СПИСОК ИСПОЛЬЗОВАННЫХ ИСТОЧНИКОВ}
\bibliographystyle{utf8gost705u}  % стилевой файл для оформления по ГОСТу
\begin{thebibliography}{3}
    \makeatletter
    \def\@biblabel#1{#1. }
    
    \bibitem{frequency}
    Park S. C., Park M. K., Kang M. G. Super-resolution image reconstruction: a technical overview //IEEE signal processing magazine. – 2003. – Т. 20. – №. 3. – С. 21-36.

    \bibitem{patch}
    Wang Q., Tang X., Shum H. Patch based blind image super resolution //Tenth IEEE International Conference on Computer Vision (ICCV'05) Volume 1. – IEEE, 2005. – Т. 1. – С. 709-716.

    \bibitem{example}
    Freeman W. T., Jones T. R., Pasztor E. C. Example-based super-resolution //IEEE Computer graphics and Applications. – 2002. – Т. 22. – №. 2. – С. 56-65.

    \bibitem{mnk}
    Katsaggelos A. K. Digital image restoration. – Springer Publishing Company, Incorporated, 2012.

    \bibitem{map}
    Elad M., Feuer A. Restoration of a single superresolution image from several blurred, noisy, and undersampled measured images //IEEE transactions on image processing. – 1997. – Т. 6. – №. 12. – С. 1646-1658.

    \bibitem{pocs}
    Fan C. et al. POCS Super-resolution sequence image reconstruction based on improvement approach of Keren registration method //Sixth International Conference on Intelligent Systems Design and Applications. – IEEE, 2006. – Т. 2. – С. 333-337.
    
    \bibitem{cnn}
    Бредихин А. И. Алгоритмы обучения сверточных нейронных сетей //Вестник Югорского государственного университета. – 2019. – №. 1 (52). – С. 41-54.

    \bibitem{pav}
    Senov A. Projective approximation based quasi-Newton methods// In: Proc. of International Workshop on Machine Learning, Optimization, and Big Data. 2017. P. 29–40.

    \bibitem{senov}
    Сенов А. А. Глубокое обучение в задаче реконструкции суперразрешения изображений //Стохастическая оптимизация в информатике. – 2017. – Т. 13. – №. 2. – С. 38-57.

    \bibitem{svertka}
    Воропаева Н. В. и др. Дискретное преобразование Фурье в обработке сигналов //Самара: Изд-во" Самар. ун. – 2015. – Т. 2015.

    \bibitem{conv}
    Convolution [Электронный ресурс]. Режим доступа: https://paperswithcode.com/method/convolution (дата обращения 22.05.2023).

    \bibitem{layers}
    Прокопеня А. С., Азаров И. С. Сверточные нейронные сети для распознавания изображений. – 2020.

    \bibitem{layers2}
    Маршалко Д. А., Кубанских О. В. Архитектура свёрточных нейронных сетей //Ученые записки Брянского государственного университета. – 2019. – №. 4 (16). – С. 10-13.

    \bibitem{train3}
    Sun Y. et al. Convolutional neural network based models for improving super-resolution imaging //Ieee Access. – 2019. – Т. 7. – С. 43042-43051.

    \bibitem{train4}
    Youm G. Y., Bae S. H., Kim M. Image super-resolution based on convolution neural networks using multi-channel input //2016 IEEE 12th Image, Video, and Multidimensional Signal Processing Workshop (IVMSP). – IEEE, 2016. – С. 1-5.
    
    \bibitem{train2}
    Созыкин А. В. Обзор методов обучения глубоких нейронных сетей //Вестник Южно-Уральского государственного университета. Серия: Вычислительная математика и информатика. – 2017. – Т. 6. – №. 3. – С. 28-59.

    \bibitem{train}
    Бредихин А. И. Алгоритмы обучения сверточных нейронных сетей //Вестник Югорского государственного университета. – 2019. – №. 1 (52). – С. 41-54.

    \bibitem{residual}
    Lin H. J., Tokuyama Y., Lin Z. J. Residual learning based convolutional neural network for super resolution //image. – 2019. – Т. 11. – С. 14.

    \bibitem{uml}
    UML Component Diagrams [Электронный ресурс]. Режим доступа: https://www.uml-diagrams.org/component-diagrams.html (дата обращения 25.05.2024).

    \bibitem{metrics}
    Sara U., Akter M., Uddin M. S. Image quality assessment through FSIM, SSIM, MSE and PSNR—a comparative study //Journal of Computer and Communications. – 2019. – Т. 7. – №. 3. – С. 8-18.

    \bibitem{metrics2}
    Hore A., Ziou D. Image quality metrics: PSNR vs. SSIM //2010 20th international conference on pattern recognition. – IEEE, 2010. – С. 2366-2369.
\end{thebibliography}
